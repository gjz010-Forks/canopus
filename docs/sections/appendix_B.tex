\section{Commutative relation of canonical gates}\label{sec:appendix_B}

Herein we present detailed proof for \Cref{thm:commutation}. The \textit{if} direction is trivial, and hence we justify the \textit{only if} direction, relying on the following two lemmas.

\begin{lemma}\label{lemma:hamiltonian_exponential}
Let $A$, $B$ be two Hermitian matrices with eigenvalues in the range $[-2,2)$. If $[e^{-i\frac{\pi}{2}A},e^{-i\frac{\pi}{2}B}]=0$ then $[A,B]=0$.
\end{lemma}
\begin{proof}
This follows from the fact that compatible observables (commuting operators) can be simultaneously diagonalized. In this case, the respective unitary matrix $e^{-i\frac{\pi}{2}A}$ commutes with $e^{-i\frac{\pi}{2}B}$. Denote by $A_{\lambda}$ the eigenspace corresponding to the eigenvalue $\lambda$ of $e^{-i\frac{\pi}{2}A}$, i.e. $e^{-i\frac{\pi}{2}A} = \oplus_{\lambda} \lambda A_{\lambda}$. Then we have
\begin{align}
    \forall \vec{v} \in A_\lambda,\, e^{-i\frac{\pi}{2}B}e^{-i\frac{\pi}{2}A}\vec{v}=e^{-i\frac{\pi}{2}B}\lambda \vec{v}=\lambda e^{-i\frac{\pi}{2}B}\vec{v}=e^{-i\frac{\pi}{2}A}e^{-i\frac{\pi}{2}B}\vec{v},
\end{align}

and thus $e^{-i\frac{\pi}{2}B}\vec{v}\in A_\lambda$. Thus $A_\lambda$ is $e^{-i\frac{\pi}{2}B}$-invariant and the restriction $e^{-i\frac{\pi}{2}B}\big\rvert_{A_{\lambda}}$ of $e^{-i\frac{\pi}{2}B}$ to $A_{\lambda}$ is still unitary since it preserves inner products. Hence it is diagonalizable and we can find an orthonormal basis $w_{\lambda_1},w_{\lambda_2},\ldots,w_{\lambda_k}$ consisting of eigenvectors of $e^{-i\frac{\pi}{2}B}\big\rvert_{A_{\lambda}}$. Note that these are also eigenvectors of $e^{-i\frac{\pi}{2}A}$ (with eigenvalue $\lambda$). Following the same token as above, for each eigenspace $E_{\lambda_i}$ of $e^{-i\frac{\pi}{2}A}$, we can construct an orthonormal basis $\beta_i$ for it consisting of eigenvectors of $e^{-i\frac{\pi}{2}B}$. Finally since the eigenspaces of different eigenvalues of $e^{-i\frac{\pi}{2}A}$ are orthogonal to each other, $\beta=\cup_i\beta_i$ forms an orthonormal basis of the entire Hilbert space $\mathcal{H}_n$ consisting of the coeigenvectors of both $e^{-i\frac{\pi}{2}A}$ and $e^{-i\frac{\pi}{2}B}$.

Now let $U$ be a unitary matrix with the vectors in $\beta$ being its columns, then
\begin{align}
    \begin{aligned}
    U^\dagger e^{-i\frac{\pi}{2}A}U&=D_A\\
    U^\dagger e^{-i\frac{\pi}{2}B}U&=D_B
    \end{aligned}
\end{align}

In general, an eigenvector of $e^{-i\frac{\pi}{2}A}$ need \textit{not} be that of $A$. However, since $A$ has its eigenvalues in the range $[-2,2)$, the map
\begin{align}
    f:[-2,2)\rightarrow U(1),a\rightarrow e^{-i\frac{\pi}{2}a}
\end{align}
is injective. Consequently different eigenvalues of $A$ correspond to different eigenvalues of $e^{-i\frac{\pi}{2}A}$, and hence the eigenspaces of $e^{-i\frac{\pi}{2}A}$ and $A$ coincide. Therefore, we have that
\begin{align}
    \begin{aligned}
    U^\dagger AU&=\Sigma_A\\
    U^\dagger BU&=\Sigma_B
    \end{aligned}
\end{align}
and since $[\Sigma_A,\Sigma_B]=0$ as they are diagonal, $[A,B]=0$. We obtain the desired result.

\end{proof}


\begin{lemma}\label{lemma:xx_rotation}
Let $P_1=(a_1X_1X_2+b_1Y_1Y_2+c_1Z_1Z_2)I_3$, $P_2=I_1(a_2X_2X_3+b_2Y_2Y_3+c_2Z_2Z_3)$ with $|c_1|\le b_1\le a_1\le\frac{1}{2}$, $|c_2|\le b_2\le a_2\le\frac{1}{2}$. If $[P_1,P_2]=0$ and $P_1,P_2\ne0$, then $b_1=b_2=c_1=c_2=0$.
\end{lemma}
\begin{proof}
Consider the product $P_1P_2$. We assume for the sake of contradiction that $b_1\ne0$. Using $[X,Y]=2iZ$, $[Y,Z]=2iX$, $[Z,X]=2iY$, we expand
\begin{align*}
[P_1,P_2] &= 2i\bigl(a_1b_2\,X_1Z_2Y_3 - b_1a_2\,Y_1Z_2X_3 + b_1c_2\,Y_1X_2Z_3\bigr) -2i\bigl(a_1c_2\,X_1Y_2Z_3 + c_1a_2\,Z_1Y_2X_3 + c_1b_2\,Z_1X_2Y_3\bigr).
\end{align*}
Since the each Pauli string is linearly independent in the $8\times8$ operator basis, e.g. term $Y_1Z_2X_3$ cannot be canceled out by any other terms, contradictory to the fact that $[P_1,P_2]=0$. Hence, vanishing of $[P_1,P_2]$ requires
\begin{align*}
a_1b_2 = a_1c_2 = b_1c_2 = b_1a_2 = c_1a_2 = c_1b_2 = 0.
\end{align*}
Since $P_1,P_2\neq0$, at least $a_1,a_2$ is nonzero, leading to $b_1 = b_2=c_1=c_2=0$. 
\end{proof}

Using \Cref{lemma:hamiltonian_exponential} and \Cref{lemma:xx_rotation} above, it is straightforward to prove \Cref{thm:commutation}. We see that $\|P_1\|\le\|a_1X_1X_2I_3\|+\|b_1Y_1Y_2I_3\|+\|c_1Z_1Z_2I_3\|\le|a_1|+|b_1|+|c_1|\le\frac{3}{2}$, where $\|\cdot\|$ is the operator norm. Hence, eigenvalues of $P_1$ are in range of $[-2,2)$. Same as the eigenvalues of $P_2$. Now if $[e^{-i\frac{\pi}{2}P_1},e^{-i\frac{\pi}{2}P_2}]=0$, then we have that $[P_1,P_2]=0$ according to \Cref{lemma:hamiltonian_exponential}, and thus $b_1=b_2=c_1=c_2=0$ according to \Cref{lemma:xx_rotation}, which proves the \textit{only if} direction.

