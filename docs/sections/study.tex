\section{Case Studies}\label{sec:studies}

\subsection{QFT kernel}\label{sec:qft_study}


\begin{table}[tbp]
    \centering
    \caption{Qubit routing comparison for the QFT kernel.}
    \label{tab:qft_example}
    \begin{footnotesize}
        \begin{tabular}{|c|l|c|c|c|c|}
    \hline
    \multicolumn{2}{|c|}{\textbf{QFT kernel}} & \multicolumn{2}{c|}{\textbf{\code{qft\_6}}} & \multicolumn{2}{c|}{\textbf{\code{qft\_12}}} \\
    \hline
    Topology & Method & \#Can & Depth2Q & \#Can & Depth2Q \\
    \hline
    \multirow{3}{*}{1D Chain} & \emph{Optimal} & \emph{15} & \emph{9} & \emph{66} & \emph{21} \\
    \cline{2-6}
     & \toqm & 16 & 10 & 67 & 22 \\
    \cline{2-6}
     & \canopus & 15 & 9 & 66 & 21 \\
    \hline
    \multirow{2}{*}{2D Square} & \toqm & 21 & 13 & 100 & 39 \\
    \cline{2-6}
     & \canopus & 15 & 9 & 73 \scriptsize{(±10\%)} & 33 \scriptsize{(±10\%)}\\
    \hline
\end{tabular}
    
    \end{footnotesize}

\end{table}


\begin{figure}[tbp]
    \centering
    \subfigure[Mapping \code{qft\_6} by \toqm.]{
        \includegraphics[width=0.485\columnwidth]{figures/qft6_toqm.pdf}
    }\subfigure[Mapping \code{qft\_6} by \canopus.]{
        \includegraphics[width=0.485\columnwidth]{figures/qft6_canopus.pdf}
    }
    \caption{\ZY{Re-draw this figure} Mapping/routing comparison for the QFT kernel. For convenient visualization, only $ \CPhase $ and $ \SWAP $ gates are shown. (a) \toqm\ generates a sub-optimal mapping scheme, with 2Q depth of 10. (b) \canopus\ generates the optimal scheme in a perfect butterfly structure, with 2Q depth of 9.}
\end{figure}

\subsection{Co-exploration of routing and ISA selection}
% TODO: 什么样的电路pattern适合什么样的ISA(以及ISA实现方式)?



\subsection{Stabilizer circuit}



\subsection{Mapping on FTQC architecture}
% TODO: 比如QFT电路
