\section{Case Studies}\label{sec:studies}

\subsection{QFT kernel}\label{sec:qft_study}


\begin{table}[tbp]
    \centering
    \caption{Qubit routing comparison for the QFT kernel.}
    \label{tab:qft_example}
    \begin{footnotesize}
        \begin{tabular}{|c|l|c|c|c|c|}
    \hline
    \multicolumn{2}{|c|}{\textbf{QFT kernel}} & \multicolumn{2}{c|}{\textbf{\code{qft\_6}}} & \multicolumn{2}{c|}{\textbf{\code{qft\_12}}} \\
    \hline
    Topology & Method & \#Can & Depth2Q & \#Can & Depth2Q \\
    \hline
    \multirow{3}{*}{1D Chain} & \emph{Optimal} & \emph{15} & \emph{9} & \emph{66} & \emph{21} \\
    \cline{2-6}
     & \toqm & 16 & 10 & 67 & 22 \\
    \cline{2-6}
     & \canopus & 15 & 9 & 66 & 21 \\
    \hline
    \multirow{2}{*}{2D Square} & \toqm & 21 & 13 & 100 & 39 \\
    \cline{2-6}
     & \canopus & 15 & 9 & 73 \scriptsize{(±10\%)} & 33 \scriptsize{(±10\%)}\\
    \hline
\end{tabular}
    
    \end{footnotesize}

\end{table}


\begin{figure}[tbp]
    \centering
    \subfigure[Mapping \code{qft\_6} by \toqm.]{
        \includegraphics[width=0.485\columnwidth]{figures/qft6_toqm.pdf}
    }\subfigure[Mapping \code{qft\_6} by \canopus.]{
        \includegraphics[width=0.485\columnwidth]{figures/qft6_canopus.pdf}
    }
    \caption{\ZY{Re-draw this figure} Mapping/routing comparison for the QFT kernel. For convenient visualization, only $ \CPhase $ and $ \SWAP $ gates are shown. (a) \toqm\ generates a sub-optimal mapping scheme, with 2Q depth of 10. (b) \canopus\ generates the optimal scheme in a perfect butterfly structure, with 2Q depth of 9.}
\end{figure}



\begin{figure}[tbp]
    \centering
    \includegraphics[width=\columnwidth]{figures/qft_cloud.pdf}
    \caption{QFT kernel fidelity comparison benchmarked on IBM\textsuperscript{®} Quantum Platform (\code{ibm\_torino}).}
    \label{fig:qft_cloud}
\end{figure}






\subsection{QEC stabilizer circuit}


\begin{figure}[tbp]
    \centering
    \includegraphics[width=\linewidth]{figures/better_check.pdf}
    \caption{Stabilizer check circuit with less routing overhead.}
    \label{fig:stabilizer}
\end{figure}

Hardware implementation of quantum LDPC (QLDPC) codes~\cite{breuckmann2021quantum, panteleev2021degenerate} remains highly challenging due to the frequent long-range interactions between qubits~\cite{bravyi2024high,wang2025demonstration}. Although emerging platforms such as neutral-atom~\cite{lin2025reuse, wang2024q, viszlai2023matching, pecorari2025high, xu2024constant} and ion-trap~\cite{wu2025boss, bruzewicz2019trapped} have shown greater potential for realizing QLDPC codes with less routing overhead, \canopus\ shows that the long-range interaction overhead can be suppressed significantly when combining the iSWAP and CX gates together --- our stabilizer ISA (\texttt{Stab-ISA}), marking an initial attempt at realizing QLDPC codes on superconducting platforms subject to topological constraints. A similar observation was also employed in~\cite{zhou2024halma} to handle defects, while it relied heavily on manual design and experience.

Figure~\ref{fig:stabilizer} shows why the \texttt{Stab-ISA} can provide optimization space for FTQC circuit execution: One ancilla qubit (grey vertex) needs to interact with multiple data qubits (blue vertices) and be measured in the end to get the syndrome information. Performing an iSWAP gate makes it possible to conduct one CX and SWAP gate together, since the iSWAP gate can be decomposed into a CX, a SWAP, and some single-qubit rotations. This allows us to ``piggyback'' a SWAP on a CX without incurring an extra two-qubit gate, enabling ancilla movement across the lattice at no additional routing cost. Therefore, \emph{qubit 3} in Figure~\ref{fig:stabilizer} can be switched to \emph{position 2} "freely" without an additional SWAP gate, thus reducing the circuit depth from $3$ to $2$.

We also build up a fully end-to-end evaluation pipeline using Stim~\cite{Gidney2021StimAF} and qLDPC~\cite{perlin2023qldpc} libraries to construct complete QLDPC code memory circuits under the circuit level noise model~\cite{Dennis_2002, Acharya2022SuppressingQE} and then decode the error syndrome using the BPOSD decoder~\cite{panteleev2021degenerate, roffe2020decoding, hillmann2024localized} to acquire the logical accuracy. We benchmark the circuit duration and logical error rate for QLDPC code examples chosen from~\cite{wang2025demonstration, panteleev2021degenerate} in Figure~\ref{fig:duration_square} \ref{fig:duration_hhex} ~\ref{fig:ler_square} ~\ref{fig:ler_hhex}, with different kinds of ISAs including \texttt{CX}, \texttt{Stab} and hardware topologies including \texttt{Square}, \texttt{Hhex}. Note that \canopus\ with \texttt{Stab-ISA} achieves the lowest circuit duration and logical error rate, among all tested examples. The reductions of circuit duration and logical error further highlight the effectiveness of native compilation of iSWAP and CX gates in mitigating mapping and routing overhead for experimental QLDPC demonstrations.


\begin{figure}[tbp]
    \centering
    \subfigure[Relative Logical Error Rate - Square Topo]{
        \includegraphics[width=\linewidth]{figures/ler_topo_square.pdf}
        \label{fig:ler_square}
    }
    \subfigure[Relative Logical Error Rate - Hhex Topo]{
        \includegraphics[width=\linewidth]{figures/ler_topo_hhex.pdf}
        \label{fig:ler_hhex}
    }
    \caption{Logical error rate comparison for stabilizer circuits mapped on square (a) and heavy-hex (b) topologies.}
\end{figure}

