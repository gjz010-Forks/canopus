\section{\canopus\ framework}\label{sec:canopus}


\subsection{Overview}


\begin{figure*}[t]
    \centering
    \includegraphics[width=\textwidth,trim={0, 0.5cm, 0, 0}, clip]{figures/profile.pdf}
    \caption{Overview of the \canopus\ framework. ...}
    \label{fig:overview}
\end{figure*}



\subsection{2Q synthesis cost modeling}\label{sec:2q_synthesis_cost_modeling}



\begin{figure*}[t]
    \centering
    \subfigure[Coverage set for \CXISA\ ISA.]{
        \includegraphics[width=\columnwidth]{figures/coverage/coverage_cx_minimal.pdf}
    }
    \subfigure[Coverage set for \SQiSWISA\ ISA.]{
        \includegraphics[width=\columnwidth]{figures/coverage/coverage_sqisw_minimal.pdf}
    }
    \subfigure[Coverage set for \SQiSWWithMirrorISA\ ISA.]{
        \includegraphics[width=\textwidth]{figures/coverage/coverage_sqisw_with_mirror_minimal.pdf}
    }
    \caption{Coverage set examples. \CXISA\ ISA: $\{ \CX, \Uthree \}$ gate set; \SQiSWISA\ ISA: $\{ \SQiSW, \iSWAP, \Uthree \}$ gate set; \SQiSWWithMirrorISA\ ISA: $\{ \SQiSW, \iSWAP, \ECP, \CX, \Uthree \}$ gate set. Costs of Basis 2Q gates are set as $ \CX\sim 1,\, \SQiSW\sim 0.75,\, \iSWAP\sim 1.5,\, \ECP\sim 1.25$.}
    \label{fig:coverage}

\end{figure*}



\subsection{Routing in canonical form}\label{sec:routing_in_canonical_form}


% \cdot
In contrast to the regular heuristic cost function used in \sabre:
\begin{align}
    H &= \frac{1}{|F|} \sum\nolimits_{(i,j)\in F} \mathrm{dist}[i,j] + \frac{k_E}{|E|}\sum\nolimits_{(i, j)\in E}\mathrm{dist}[i,j]\notag\\
    &= \mathrm{Avg}\{\mathrm{dist}[i,j]\}_F + k_E\,\mathrm{Avg}\{\mathrm{dist}[i,j]\}_E 
    % &= \widebar{\mathrm{dist}(F)} + k_E\,\widebar{\mathrm{dist}(E)} \notag
\end{align}
which involves the basic (left term) and lookahead (right term) components.
In practice, there is a $ w_{\mathrm{decay}} $ decay factor applied to $H$, which is not shown as it does not affect the composition of $H$.


The heuristic cost function in \canopus\ is defined as:
\begin{align}
    H = w_d\, \Delta_{\mathrm{depth}} + c_0 (\Delta_{\mathrm{Avg}\{\mathrm{dist}[i,j]\}_F} + k_E \, \Delta_{\mathrm{Avg}\{\mathrm{dist}[i,j]\}_E})
\end{align}



- Unified and highly-effective qubit routing approach in canonical form, with properties of quantum ISAs tailored to the routing process


\subsection{Enhanced optimization via commutation}\label{sec:gate_commutation_guided_optimization}



\begin{figure}[tbp]
    \centering
    \includegraphics[width=\columnwidth]{figures/commutation_relation.pdf}
    \caption{Canonical gate representation enables easily capturing commutative relations within real-world circuits.}
    \label{fig:commutation}
    
\end{figure}



- Capture optimization opportunities exposed by gate commutation; while commutation relations can be uniformly described in canonical form



\begin{theorem}[Canonical gate commutation]\label{thm:commutation}
    Let $\,\Can(a,b,c)_{q_0,q_1}$ and $\,\Can(a',b',c')_{q_1,q_2}$ denote canonical gates acting on qubits ($q_0, q_1$) and ($q_1, q_2$) respectively, with an overlapping qubit $ q_1 $. They are commutative if and only if
    \begin{align}
        b=b'=c=c'=0,
    \end{align}
    that is, when both consist solely of $\,\XX$ rotations.
\end{theorem}






\ZY{Proposition?? Theorem?}




% \subsection{Qubit dependencies guided optimization}\label{sec:qubit_dependencies_guided_optimization}
% - Capture optimization opportunities exposed by qubit dependencies, which implies optimization in a more global scope




