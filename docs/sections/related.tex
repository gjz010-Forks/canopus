\section{Related Works}\label{sec:related}

Qubit mapping/routing is one the the most well-explored topic of quantum compiler research, as it shares the similar methodologies with instruction scheduling~\cite{codina2001unified,hennessy1983postpass} and register allocation~\cite{chaitin1982register,poletto1999linear} in classical computing. Conventional methods focus on the simplified routing model, that is, \#$ \SWAP $-minimal insertion, three-$ \CX $-unrolled $ \SWAP $ gate, and $ \CX $-based latency metric. That brings a gap between quantum hardware performance and its ceiling, which is particularly evident with the progress of underlying instruction models for modern quantum hardware.


\citet{zulehner2018efficient} introduces an A*-based algorithm to minimize SWAP gate overhead for concurrent CNOT gate layers. The approach partitions the circuit into layers and solves the mapping problem subsequently. \citet{li2019tackling} also utilizes the circuit DAG layering thought to tackle the qubit mapping problem and proposes the bidirectional routing procedure to acquire a better initial mapping desired to result in \#$ \SWAP $ inserted minimization as expected. It also briefly discusses the trade-off between the inserted $ \SWAP $ count and the circuit depth but does not prioritize optimizing circuit depth. Some other works leverage algorithmic procedures similar to \sabre\ to improve parallelism among inserted $\SWAP$s and other 2Q gates~\cite{lao2021timing,ddroute2025,zou2024lightsabre}, or attempt to minimize circuit depth via graph matching~\cite{childs2019circuit}. \citet{zhang2021time} systematically investigates the time (circuit depth) optimality of qubit mapping and proposed an A*-based method \toqm\ that results in better results than the SOTA solver-based depth-driven algorithm~\cite{tan2020optimal}. However, the optimality of qubit routing is a complex task. There are rarely theoretical studies that claims the holistic optimality of some $ \SWAP $ insertion schemes provided the quantum ISAs, device topologies, and synthesis cost models. In our field tests, TOQM does not lead to time-optimal results compared to our heuristic \canopus, and the optimal mapping scheme for specific patterns such as QFT kernel analyzed in \cite{zhang2021time} are not indeed optimal, according to our case study in \Cref{sec:qft_study}.

With the recent development of advanced quantum ISAs such as superconducting fractional gates~\cite{ibmFractionalGates}, ion-trapped partial entangling gates~\cite{ionqPartialGates,yale2025realization}, and the AshN scheme~\cite{chen2024one}, some works began exploring how to efficiently utilize these ISAs to make compiler optimizations closer to hardware characteristics. \citet{mckinney2024mirage} investigates the practical performance of \SQiSWISA\ ISA proposed by \citet{huang2023quantum} and the synthesis capability when incorporating the basis gates' mirrors into the ISA. The modified \sabre\ algorithm in \cite{mckinney2024mirage} provides an attempt of the collaborative gate decomposition and qubit routing approach, while the optimization opportunities considered therein are limited and the algorithmic techniques are not sophisticated. \bqskit~\cite{bqskit} and the series of works behind~\cite{davis2019heuristics,wu2020qgo,kukliansky2023qfactor,younis2021qfast} provides a toolkit to rebase arbitrary 2Q unitaries to specific ISAs through approximate synthesis (structural search and numerical optimization) that is not computational efficient. Approximate synthesis by \bqskit does not ensure an optimal schemes for two-qubit and multi-qubit synthesis cases. In addition, due to the lack of native compilation strategies and rational synthesis cost model, \citet{kalloor2024quantum} claims that alternative ISAs are hard to be comparable to \CXISA\ when evaluating quantum hardware roofline by \bqskit. As for applicability of expanded ISAs to QEC, Google's latest theoretical~\cite{mcewen2023relaxing} and experimental~\cite{eickbusch2024demonstrating} works demonstrate the \CXISA-\iSWAPISA\ combination ISA could benefits suppressing fault-tolerant threshold. \citet{zhou2024halma} proposes a routing-based method enhanced by \CXISA-\iSWAPISA\ for overcoming ancilla defects among surface code blocks while preserving encoded logical information.



% - expanded ISAs \& systematic utilization
%     - Noise-aware ...
%     - Mirage ... not sophisticated algorithm, ... a subset of our approach
%     - BQSKit ... quantum hardware roofline .. they claim that ... 
%     - The last-step \dquote{synthesizer} .... most based on \dquote{approximate synthesis} (roofline, heterogeneous, ZZ(theta))

% (((((((None of them make deep co-optimization tailored to various quantum ISAs in a systematic and efficient approach)))))))


% With respect to the heuristic cost for $ \SWAP $ search, our routing algorithm involves the duration (generalization metric of circuit depth) driven goal by taking the canonical gate synthesis cost into the circuit duration increment. 



