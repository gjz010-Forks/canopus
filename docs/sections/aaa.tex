
Qubit mapping is a critical compilation stage for both near-term and fault-tolerant quantum computers, yet scalable routing methods often incur significant overhead in circuit depth and duration. This performance gap stems from a fundamental disconnect: compilers rely on simplified models, like SWAP gate insertion, which fail to exploit the native gate properties of advanced hardware. Consequently, the true potential of diverse and powerful instruction set architectures (ISAs) remains untapped.

Recent hardware breakthroughs have enabled high-fidelity implementations of advanced ISAs, such as those based on 
\SQiSW
 or 
\ZZ
(
θ
)
 gates, which offer superior synthesis efficiency and inherent noise resilience. However, the absence of a systematic, ISA-aware compilation framework has prevented the community from leveraging their full capabilities.

To bridge this gap, we introduce Canopus, a unified routing framework designed to co-optimize circuit synthesis and qubit mapping across diverse ISAs. By leveraging the canonical representation of two-qubit gates and monodromy polytope theory, Canopus accurately models synthesis costs, enabling a more intelligent search for routing solutions. Furthermore, our framework formalizes commutation rules through this representation, unlocking a generalized approach to gate reordering optimizations. Our experiments demonstrate that Canopus consistently reduces routing overhead by 30-40% compared to state-of-the-art methods across various ISAs and hardware topologies. This work presents the first coherent methodology for the co-exploration of program structure, ISA choice, and hardware connectivity, paving the way for more effective quantum software-hardware co-design.
